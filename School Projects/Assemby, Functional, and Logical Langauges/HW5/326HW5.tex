\documentclass[12pt]{article}
\usepackage[utf8]{inputenc}
\usepackage[margin = 16mm]{geometry}

\begin{document}

\begin{center}

\Huge
Bryan Kline\\
[10mm]
CS326\\ 
[10mm]
Homework 5\\
\small LaTex (TeXstudio)\\
[10mm]
\Huge
11/10/2016\\
[200mm]

\end{center}

\begin{flushleft}

1. The Boolean type is typically represented on a byte in memory, although it can have only two possible values. Why not represent the Boolean type on a single bit?\\
[2mm]

\qquad Addresses refer to whole bytes in memory and so eight bits is in a sense the smallest addressable\\
\qquad length of memory in most computer architectures, and while it would in principle be possible\\
\qquad to address individual bits, doing so would take up more space in terms of the number of\\
\qquad addresses necessary to keep, eight times as many.  It's more efficient to simply give Boolean type\\
\qquad variables an entire byte to save from having to address to the bit level.  There is also the question\\
\qquad of type compatibility; in languages like C where non-Boolean types can be treated like Booleans\\
\qquad and where Booleans can have some arithmetic operations carried out on them, it's important\\
\qquad that they be a byte to facilitate this.\\
[4mm]


2. Write a small fragment of code that shows how unions can be used in C to interpret the bits of a value of one type as if they represented a value of some other type (non-converting type cast).\\
[2mm]

{\fontfamily{qcr}\selectfont
\qquad \qquad \textbf{union}\\
\qquad \qquad \{\\
\qquad \qquad \qquad \textbf{int} x;\\
\qquad \qquad \qquad \textbf{float} y;\\
\qquad \qquad \} example;\\
[2mm]    
\qquad \qquad //the number 23 in IEEE 754 floating point representation\\
\qquad \qquad example.x = 0b01000001101110000000000000000000;\\
[2mm]   

\qquad \qquad printf("\%f\textbackslash n", example.y);\\
\qquad \qquad printf("\%d\textbackslash n", example.x);\\
[2mm]   

\qquad \qquad example.x = example.x / 2;\\
\qquad \qquad example.y = example.y + 0.2;\\
[2mm]   
\qquad \qquad printf("\textbackslash n");\\
\qquad \qquad printf("\%f\textbackslash n", example.y);\\
\qquad \qquad printf("\%d\textbackslash n", example.x);\\[2mm]
}

\qquad \qquad The above outputs the following:\\
[2mm]

{\fontfamily{qcr}\selectfont
\qquad \qquad \qquad 23.000000\\
\qquad \qquad \qquad 1102577664\\
[2mm]

\qquad \qquad \qquad 0.200000\\
\qquad \qquad \qquad 1045220557\\
[2mm]
}

\qquad \qquad The union creates space for the largest element in the union, in this case they are the same\\
\qquad \qquad size, but space for only is allocated and in the first line where {\fontfamily{qcr}\selectfont example.x} is set to the bit\\
\qquad \qquad string corresponding to the IEEE 754 floating point representation of the number 23, the\\
\qquad \qquad value of {\fontfamily{qcr}\selectfont example.x} is $1102577664$ and the value of {\fontfamily{qcr}\selectfont example.y} is 23.000000.  When the int\\
\qquad \qquad is divided by two, and the float has 0.2 added to it their values both change in ways that\\
\qquad \qquad correspond to adding 0.2 to the IEEE representation.  This demonstrates that the union\\
\qquad \qquad creates space for only one variable and then any changes to either name in the union hold\\
\qquad \qquad for the other.\\
[30mm] 




3. Consider the following:\\

\qquad {\fontfamily{qcr}\selectfont

\qquad \textbf{typedef struct}\\
\qquad \{\\
\qquad \qquad \textbf{int} x;\\
\qquad \qquad \textbf{char} y;\\
\qquad \} Rec1;\\
[2mm]
\qquad \textbf{typedef} Rec1 Rec2;\\
[2mm]
\qquad \textbf{typedef struct}\\
\qquad\{\\
\qquad \qquad \textbf{int} x;\\
\qquad \qquad \textbf{char} y;\\
\qquad \} Rec3;\\
[2mm]
\qquad Rec1 a, b;\\
\qquad Rec2 c;\\
\qquad Rec3 d;\\
[4mm]
}

Specify which of the variables {\fontfamily{qcr}\selectfont a,b,c,d } are type equivalent under (a) structural equivalence, (b) strict name equivalence, and (c) loose name equivalence.\\
[2mm]

\qquad a)\\
\qquad \qquad The variables {\fontfamily{qcr}\selectfont a,b} and {\fontfamily{qcr}\selectfont d} are structurally equivalent because {\fontfamily{qcr}\selectfont a} and {\fontfamily{qcr}\selectfont b} are both of type {\fontfamily{qcr}\selectfont Rec1}\\
\qquad \qquad and {\fontfamily{qcr}\selectfont d} is of type {\fontfamily{qcr}\selectfont Rec3} and {\fontfamily{qcr}\selectfont Rec1} and {\fontfamily{qcr}\selectfont Rec3} are structurally the same, i.e. have the same\\
\qquad \qquad  fields in the same order.\\
[2mm]

\qquad b)\\
\qquad \qquad The variables {\fontfamily{qcr}\selectfont a} and {\fontfamily{qcr}\selectfont b} are strictly name equivalent because they are both of type {\fontfamily{qcr}\selectfont Rec1}, and\\
\qquad \qquad not name equivalent with {\fontfamily{qcr}\selectfont c} because {\fontfamily{qcr}\selectfont \textbf{typedef} Rec1 Rec2;} is a declaration and a\\
\qquad \qquad definition where {\fontfamily{qcr}\selectfont c} is a different type.\\
[2mm]

\qquad c)\\
\qquad \qquad The variables {\fontfamily{qcr}\selectfont a,b} and {\fontfamily{qcr}\selectfont c} are loosely name equivalent because they are both of type {\fontfamily{qcr}\selectfont Rec1},\\
\qquad \qquad the line {\fontfamily{qcr}\selectfont \textbf{typedef} Rec1 Rec2;} makes {\fontfamily{qcr}\selectfont Rec2} and alias of {\fontfamily{qcr}\selectfont Rec1}, just a different name for\\
\qquad \qquad the same type.\\
[4mm]

4. Consider the following program, written in C:\\
[2mm]

{\fontfamily{qcr}\selectfont
\qquad \textbf{typedef struct}\\
\qquad \{\\
\qquad \qquad \textbf{int} x;\\
\qquad \qquad \textbf{int} y;\\
\qquad \} Foo;\\
[2mm]
\qquad \textbf{void} allocate\_node (Foo * f)\\
\qquad \{\\
\qquad \qquad f = (Foo *) malloc ( sizeof(Foo) );\\
\qquad \}\\
[2mm]
\qquad \textbf{void} main()\\
\qquad \{\\
\qquad \qquad Foo * p;\\
\qquad \qquad allocate\_node (p);\\
\qquad \qquad p->x = 2;\\
\qquad \qquad p->y = 3;\\
\qquad \qquad free(p);\\
\qquad \}\\
[2mm]
}

Although the program compiles, it produces a run-time error. Why?  Rewrite the two functions allocate\_node and main so that the program runs correctly.\\
[2mm]

\qquad \qquad The reason that it produces a run-time error is that when {\fontfamily{qcr}\selectfont p} is dereferenced and dotted to\\
\qquad \qquad assign the fields {\fontfamily{qcr}\selectfont x} and {\fontfamily{qcr}\selectfont y} the pointer {\fontfamily{qcr}\selectfont p} doesn't point to anything and that causes a\\
\qquad \qquad segmentation fault.  The reason that {\fontfamily{qcr}\selectfont p} doesn't point to anything is that it's not changed in\\
\qquad \qquad the function.  In order to change {\fontfamily{qcr}\selectfont p} a pointer to it must be passed in, or a pointer to a {\fontfamily{qcr}\selectfont Foo}\\
\qquad \qquad pointer.  The code below, which has a print statement added to verify that the struct fields\\
\qquad \qquad were changed, demonstrates this:\\
[2mm]

{\fontfamily{qcr}\selectfont 

\qquad \qquad \textbf{typedef struct}\\ 
\qquad \qquad \{\\
\qquad \qquad \qquad \textbf{int} x;\\
\qquad \qquad \qquad \textbf{int} y;\\
\qquad \qquad \} Foo;\\
[2mm]
\qquad \qquad \textbf{void} allocateNode(Foo** f)\\
\qquad \qquad \{\\
\qquad \qquad \qquad *f = (Foo*) malloc(\textbf{sizeof}(Foo));\\
\qquad \qquad \}\\
[2mm]
\qquad \qquad \textbf{int} main()\\
\qquad \qquad \{\\
\qquad \qquad \qquad Foo **p;\\
\qquad \qquad \qquad Foo* z;\\
[2mm]	
\qquad \qquad \qquad allocateNode(p);\\
[2mm]
\qquad \qquad \qquad z = *p;\\
\qquad \qquad \qquad z->x = 2;\\
\qquad \qquad \qquad z->y = 3;\\
[2mm]	
\qquad \qquad \qquad printf("x is \%d and y is \%d\textbackslash n", z->x, z->y);\\
[2mm]	
\qquad \qquad \qquad free(*p);\\
\qquad \qquad \qquad p = NULL;\\	
\qquad \qquad \qquad z = NULL;\\
[2mm]
\qquad \qquad \qquad \textbf{return} 0;\\
\qquad \qquad \}\\[4mm]
	
}

\qquad \qquad The above code outputs:\\
[2mm]
\qquad \qquad \qquad {\fontfamily{qcr}\selectfont x is 2 and y is 3}\\
[2mm]
\qquad \qquad So the struct fields were successfully changed because a pointer to a {\fontfamily{qcr}\selectfont Foo} pointer was passed\\
\qquad \qquad into the function wherein memory was allocated for a {\fontfamily{qcr}\selectfont Foo} struct so that when it's\\
\qquad \qquad dereferenced it doesn't seg fault.\\
[10mm]

5. (\textbf{Extra Credit}) A compiler for a language with static scoping typically uses a LeBlanc-Cook symbol table to track the bindings of various names encountered in a program. Describe the mechanism that should be used by a compiler to ensure that the names of record fields used inside a with statement are bound to the correct objects. Specify what should be placed in (a) the symbol table, (b) the scope stack, and (c) the run-time program stack.\\
[2mm]




\end{flushleft}

\end{document}