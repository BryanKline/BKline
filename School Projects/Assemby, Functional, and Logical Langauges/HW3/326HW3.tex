\documentclass[12pt]{article}

\usepackage[margin = 16mm]{geometry}

\begin{document}

\begin{center}

\Huge
Bryan Kline\\
[10mm]
CS326\\ 
[10mm]
Homework 3\\
\small LaTex (TeXstudio)\\
[10mm]
\Huge
10/11/2016\\
[200mm]

\end{center}

\begin{flushleft}

1. Using the C++ programming language, indicate the binding time (language design, language implementation, compilation, link, run, etc.) for each of the following attributes.  Justify your answer.\\
[2mm]

\qquad a) The variable declaration that corresponds to a certain variable reference (use)\\ 
\qquad \qquad This takes place during compile time, as C and C++ are statically scoped.\\
\qquad b) The range of possible values for integer numbers\\ 
\qquad \qquad This is determined at during the language implementation.\\
\qquad c) The meaning of char\\
\qquad \qquad This is determined when the language is initially designed.\\
\qquad d) The address of a local variable\\ 
\qquad \qquad This depends on which address is meant, if it's the physical address then it's during\\
\qquad \qquad load time, and if it's the virtual address then it's during compile time.\\
\qquad e) The address of a library function\\ 
\qquad \qquad This occurs at link time.\\
\qquad f) The referencing environment of a function passed as a parameter\\
\qquad \qquad This is done at compile time, as C and C++ don't have nested functions.\\ 
\qquad g) The total amount of memory needed by the data in a program\\
\qquad \qquad This is determined at runtime.\\
[4mm]

2.  Can a language that uses dynamic scoping do type checking at compile time? Why?  Can a language that uses static scoping do type checking at run time? Why?\\
[2mm]

\qquad No, a language that uses dynamic scoping cannot do type checking at compile time.  This is\\
\qquad because if the scope is determined dynamically, and if the flow of control of a program isn't\\
\qquad something that can be predicted ahead of time, i.e. what functions might be called in what\\
\qquad order, then types cannot be checked because it won't know what to expect.  Also, a language that\\
\qquad uses static scoping cannot do type checking at runtime because that is all done at compile time.\\
[2mm]
 

3.  Does Scheme use static or dynamic scoping? Write a short Scheme program that
proves your answer.\\
[2mm]
\qquad Scheme uses static scoping, as the textbook for this class points out (p. 140) as a footnote to\\
\qquad listings dynamically scoped languages.  The following Scheme code snippet will show this:\\
[2mm]

\qquad (define x 0)\\

\qquad (define (function)\\
\qquad \qquad \qquad (set! x 1))\\
[2mm]
\qquad (set! x 2)\\
\qquad (function)\\
\qquad (display x)\\
[2mm]

\qquad The output is 1, not 2, which demonstrates that the scoping is static because the global x is set\\
\qquad to 1 by the call to function, meaning that the next scope out from function, the global scope, was\\
\qquad changed.\\
[2mm]

4.  Consider the following pseudo-code:\\
[2mm]

\qquad x : integer; \qquad - - global\\
\qquad procedure set\_x (n : integer)\\
\qquad x := n;\\
\qquad procedure print\_x\\
\qquad \qquad write\_integer (x);\\
\qquad procedure foo (S, P : procedure; n : integer)\\
\qquad \qquad x : integer;\\
\qquad \qquad if n in {1,3}\\
\qquad \qquad \qquad set\_x(n);\\
\qquad \qquad else\\
\qquad \qquad \qquad S(n);\\
\qquad \qquad if n in {1,2}\\
\qquad \qquad \qquad print\_x;\\
\qquad \qquad else\\
\qquad \qquad \qquad P;\\
\qquad - - main program\\
\qquad set\_x(0); foo (set\_x, print\_x, 1); print\_x;\\
\qquad set\_x(0); foo (set\_x, print\_x, 2); print\_x;\\
\qquad set\_x(0); foo (set\_x, print\_x, 3); print\_x;\\
\qquad set\_x(0); foo (set\_x, print\_x, 4); print\_x;\\
[2mm]

Assume that the language uses dynamic scoping. What does this program print if the
language uses shallow binding? Why? What does it print with deep binding? Why?
Note: At exactly one point during execution in the deep binding case, the program will
attempt to print an uninitialized variable. Simply write a “?” for the value printed at that
point.\\
[2mm]

\qquad NOTE: Newline characters, while never printed in the code, are added for each line for clarity.\\
[2mm]

\qquad Shallow binding:\\
[2mm]

\qquad \qquad 1 0\\
\qquad \qquad 2 0\\
\qquad \qquad 3 0\\
\qquad \qquad 4 0\\

\qquad The above is printed out if the binding is shallow because the reference environment comes from\\
\qquad whenever the function is called, in this case when set\_x and print\_x are called in foo, not from\\
\qquad when they were passed into foo, so they print the x from inside foo.\\
[2mm]

\qquad Deep binding:\\
[2mm]

\qquad \qquad 1 0\\
\qquad \qquad ? 2\\
\qquad \qquad 0 0\\
\qquad \qquad 4 4\\

\qquad The above is printed out if the binding is deep because the reference environment comes from\\
\qquad when the functions are passed into foo, and so if in foo the local set\_x and print\_x functions are\\
\qquad called then the local x is printed, otherwise it's the global one, and when it leaves foo it's also the\\
\qquad global x that's printed.\\

\end{flushleft}

\end{document}