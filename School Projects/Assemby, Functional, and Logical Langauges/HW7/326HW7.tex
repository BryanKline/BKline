\documentclass[12pt]{article}
\usepackage[utf8]{inputenc}
\usepackage[margin = 16mm]{geometry}

\begin{document}

\begin{center}

\Huge
Bryan Kline\\
[10mm]
CS326\\ 
[10mm]
Homework 7\\
\small LaTex (TeXstudio)\\
[10mm]
\Huge
12/01/2016\\
[200mm]

\end{center}

\begin{flushleft}

1.\\
Write the rules for a predicate {\fontfamily{qcr}\selectfont reverse(L,L1)}, which succeeds if list {\fontfamily{qcr}\selectfont L1} is the list {\fontfamily{qcr}\selectfont L} reversed. The following query shows an example of using this predicate:\\
[2mm]

{\fontfamily{qcr}\selectfont
	
\qquad ?- reverse([1,2,3], L1).\\
\qquad L1 = [3,2,1]\\
[6mm]

\qquad \qquad \%reverse:\\
\qquad \qquad reverse([], []).\\
\qquad \qquad reverse([H|T], L1) :-  reverse(T, L2), append( L2, [H], L1).\\


}

2.\\
Write the rules for a predicate {\fontfamily{qcr}\selectfont take(L,N,L1)}, which succeeds if list {\fontfamily{qcr}\selectfont L1} contains the first {\fontfamily{qcr}\selectfont N} elements of list {\fontfamily{qcr}\selectfont L}, in the same order. The following queries show examples of using this predicate:\\
[2mm]

{\fontfamily{qcr}\selectfont

\qquad ?- take([5,1,2,7], 3, L1).\\
\qquad L1 = [5,1,2]\\
\qquad ?- take([5,1,2,7], 10, L1).\\
\qquad L1 = [5,1,2,7]\\
[6mm]

}

	
\qquad NOTE: I made another function to return the number of elements in a list so that if the number\\
\qquad given to {\fontfamily{qcr}\selectfont take} is larger than the size of the list it just returns the list.\\
[2mm]
{\fontfamily{qcr}\selectfont
	
\qquad \qquad \%numElements:\\	
\qquad \qquad numElements([], 0).\\
\qquad \qquad numElements([\_|T], X) :- numElements(T, X1), X is X1 + 1.\\
[2mm]
\qquad \qquad \%take:\\
\qquad \qquad take(\_, 0, []).\\
\qquad \qquad take(L1, N, L1) :- numElements(L1, X), X =< N.\\
\qquad \qquad take([H|T], N, L1) :- X is N - 1, take(T, X, L2),\\
\qquad \qquad \qquad \qquad \qquad \qquad \qquad \qquad \qquad  \qquad \qquad \qquad \qquad append([H], L2, L1).\\
}

3.\\
Consider the following definition of a binary tree in Prolog, where a tree is either the constant {\fontfamily{qcr}\selectfont nil}, or a structure {\fontfamily{qcr}\selectfont node} with 3 elements, the second and third elements also being trees:\\
[2mm]

{\fontfamily{qcr}\selectfont
	
\qquad tree(nil).\\
\qquad tree(node(\_,Left,Right)) :- tree(Left), tree(Right).\\
[4mm]

}

\qquad a)\\
\qquad Write the rules for a predicate {\fontfamily{qcr}\selectfont nleaves(T,N)}, which succeeds if {\fontfamily{qcr}\selectfont N} is the number
of leaves in\\
\qquad the tree {\fontfamily{qcr}\selectfont T}. The following query shows an example of using this predicate:\\
[2mm]
{\fontfamily{qcr}\selectfont
	
\qquad \qquad ?- nleaves(node(1, node(2, node(3,nil,nil), node(4,nil,nil)),\\
\qquad \qquad node(5,nil,nil)), N).\\
\qquad \qquad N = 3\\
[6mm]

\qquad \qquad \qquad \%nleaves:\\
\qquad \qquad \qquad nleaves(node(\_, nil, nil), N) :- N is 1.\\
\qquad \qquad \qquad nleaves(node(\_, Left, Right), N) :- nleaves(Left, N2),\\
\qquad \qquad \qquad \qquad \qquad \qquad \qquad \qquad \qquad nleaves(Right, N3), N is N2 + N3.\\
[6mm]
}

\qquad b)\\
\qquad Write the rules for a predicate {\fontfamily{qcr}\selectfont treeMember(E,T)}, which succeeds if {\fontfamily{qcr}\selectfont E} appears as an element in\\
\qquad the tree {\fontfamily{qcr}\selectfont T}. The following query shows an example of using this predicate:\\
[2mm]
{\fontfamily{qcr}\selectfont
\qquad \qquad ?- treeMember(3, node(1, node(2, node(3,nil,nil),\\
\qquad \qquad node(4,nil,nil)), node(5,nil,nil))).\\
\qquad \qquad Yes\\
[6mm]

\qquad \qquad \qquad \%treeMember:\\
\qquad \qquad \qquad treeMember(E, node(E, \_, \_)).\\
\qquad \qquad \qquad treeMember(E, node(\_, Left, \_)) :- treeMember(E, Left).\\
\qquad \qquad \qquad treeMember(E, node(\_, \_, Right)) :- treeMember(E, Right).\\
[2mm]
}


\qquad c)\\
\qquad Write the rules for a predicate {\fontfamily{qcr}\selectfont preOrder(T,L)}, which succeeds if {\fontfamily{qcr}\selectfont L} is a list containing all\\
\qquad elements in the tree {\fontfamily{qcr}\selectfont T} corresponding to a pre-order traversal of the tree. The following query\\
\qquad shows an example of using this predicate:\\
[2mm]
{\fontfamily{qcr}\selectfont
\qquad \qquad ?- preOrder(node(1, node(2, node(3,nil,nil), node(4,nil,nil)),\\
\qquad \qquad node(5,nil,nil)), L).\\
\qquad \qquad L = [1, 2, 3, 4, 5]\\
[6mm]	
	
\qquad \qquad \qquad \%preOrder:\\	
\qquad \qquad \qquad preOrder(node(N, nil, nil), [N]).\\
\qquad \qquad \qquad preOrder(node(H, Left, nil), [H|L]) :-\\
\qquad \qquad \qquad\qquad \qquad \qquad\qquad \qquad preOrder(Left, L1),
 append(L1, [], L).\\
\qquad \qquad \qquad preOrder(node(H, nil, Right), [H|L]) :-\\ 
\qquad \qquad \qquad\qquad \qquad \qquad\qquad \qquad preOrder(Right, L1), append([], L1, L).\\
\qquad \qquad \qquad preOrder(node(H, Left, Right), [H|L]) :-\\
\qquad \qquad \qquad \qquad \qquad \qquad preOrder(Left, L1), preOrder(Right, L2),\\
\qquad \qquad \qquad \qquad \qquad \qquad \qquad \qquad \qquad \qquad \qquad \qquad \qquad append(L1, L2, L).\\
[2mm]
}

\qquad d)\\
\qquad The \textit{height} of a tree is defined as the maximum number of nodes on a path from the root to a\\
\qquad leaf. Write the rules for a predicate {\fontfamily{qcr}\selectfont height(T,N)}, which succeeds if {\fontfamily{qcr}\selectfont N} is the height of the tree\\
\qquad {\fontfamily{qcr}\selectfont T}.  The following query shows an example of using this predicate:\\
[2mm]
{\fontfamily{qcr}\selectfont
\qquad \qquad ?- height(node(1, node(2, node(3,nil,nil), node(4,nil,nil)),\\
\qquad \qquad node(5,nil,nil)), N).\\
\qquad \qquad N = 3\\
[6mm]	


\qquad \qquad \qquad \%height:\\	
\qquad \qquad \qquad height(node(\_, nil, nil), N) :- N is 1.\\
\qquad \qquad \qquad height(node(\_, Left, nil), N) :-\\
\qquad \qquad \qquad \qquad \qquad \qquad \qquad \qquad \qquad \qquad height(Left, N1), N is N1 + 1.\\
\qquad \qquad \qquad height(node(\_, nil, Right), N) :-\\
\qquad \qquad \qquad \qquad \qquad \qquad \qquad \qquad \qquad \qquad height(Right, N1), N is N1 + 1.\\
\qquad \qquad \qquad  preOrder(node(H, Left, Right), [H|L]) :-\\
\qquad \qquad \qquad \qquad \qquad \qquad preOrder(Left, L1), preOrder(Right, L2),\\
\qquad \qquad \qquad \qquad \qquad \qquad \qquad \qquad \qquad \qquad \qquad \qquad \qquad append(L1, L2, L).
}

\qquad You may want to use the predefined arithmetic function {\fontfamily{qcr}\selectfont max(X,Y)}.\\
[4mm]

4.\\
Write the rules for a predicate {\fontfamily{qcr}\selectfont insert(X,L,L1)}, which succeeds if list {\fontfamily{qcr}\selectfont L1} is
identical to the sorted list {\fontfamily{qcr}\selectfont L} with {\fontfamily{qcr}\selectfont X} inserted at the correct place. Assume that {\fontfamily{qcr}\selectfont L} is already sorted. The following query shows an example of using this predicate:\\
[2mm]

{\fontfamily{qcr}\selectfont
\qquad ?- insert(5, [1,3,4,7], L1).\\
\qquad L1 = [1,3,4,5,7]\\
[6mm]

\qquad \qquad \%insert:\\
\qquad \qquad insert(N, [], [N]).\\
\qquad \qquad insert(N, [H|T], L) :- N < H,  append([N], [H|T] , L).\\
\qquad \qquad insert(N, [H|T], L) :- N >= H, insert(N, T, L1),\\
\qquad \qquad \qquad \qquad \qquad \qquad \qquad \qquad \qquad \qquad \qquad \qquad append([H], L1, L).\\
[6mm]
}


5. Extra Credit\\
Write the rules for a predicate {\fontfamily{qcr}\selectfont flatten(A,B)}, which succeeds if {\fontfamily{qcr}\selectfont A} is a
list (possibly containing sublists), and {\fontfamily{qcr}\selectfont B} is a list containing all elements in {\fontfamily{qcr}\selectfont A} and its sublists, but all at the same level. The following query shows an example of using this predicate:\\
[2mm]

{\fontfamily{qcr}\selectfont
	
\qquad ?- flatten([1, [2, [3, 4]], 5], L).\\
\qquad L = [1, 2, 3, 4, 5]\\
[6mm]

\qquad \qquad \%flatten:\\
\qquad \qquad flatten([H|[]], [H]).\\
\qquad \qquad flatten([[H|T1]|T2], L) :- flatten([H], L4),\\
\qquad \qquad \qquad \qquad \qquad \qquad flatten(T1, L1), flatten(T2, L2),\\
\qquad \qquad \qquad \qquad \qquad \qquad \qquad append(L1, L2, L3), append(L4, L3, L).\\
\qquad \qquad flatten([H|T], L) :- flatten([H], L1),\\
\qquad \qquad \qquad \qquad \qquad \qquad \qquad \qquad flatten(T, L2), append(L1, L2, L).\\
[6mm]

}

\qquad \qquad NOTE: This doesn't quite work, but it will flatten out most of a list.  It needs more work,\\
\qquad \qquad but I don't have time to finish it.\\

\end{flushleft}

\end{document}